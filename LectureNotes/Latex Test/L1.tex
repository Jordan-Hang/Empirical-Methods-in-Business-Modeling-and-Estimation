\documentclass[a4paper]{article}
\usepackage{amsmath,amsfonts,amssymb}
\usepackage[utf8]{inputenc}
\usepackage{xeCJK}
\usepackage{graphicx}
\usepackage{mathtools}
\usepackage{geometry}
\usepackage{enumitem}
\usepackage{hyperref}

\begin{document}

% Customize the symbols
\setlist[itemize,2]{label=$\circ$}
\setlist[itemize,3]{label=$\triangleright$}

\title{Notes on Empirical Methods in Business \\ Preview: Introduction}
\author{Hang XU \\hxuch@connect.ust.hk}
\date{\today}
\maketitle

\section{Introduction}
During the first year of my PhD, I have taken a course on Empirical Methods in Business: Modeling and Estimation. 
This course is designed to provide students with a comprehensive understanding of the most commonly used empirical methods in business research. The main topics of the course can be seen in the following part. 
I mainly used handwritten notes when I was taking the lectures, which is hard to formalize. As a PhD researcher specializing in empirical studies, it is necessary to have a clear understanding of common empirical methods.
Therefore, I review the course content and summarize it in a more formal way to help others who are interested in empirical methods in business research.

Worth to mention that, all the faults in the notes are mine, and I will try my best to make it accurate and clear. If you find any mistakes or have any suggestions, please feel free to contact me.

\section{Main Methods Covered}
\begin{itemize}
    
    \item Issues in Regression
    \begin{itemize}
        \item Multicollinearity
        \item Heteroskedasticity
        \item Causality
    \end{itemize}
    
    \item Causal Inference and Treatment Effect Models
    \begin{itemize}
        \item Instrument Variables
        \item Panel Data with Fixed Effects
        \item Treatment Effects
        \begin{itemize}
            \item Matching
            \item Propensity Score Matching
            \item Inverse Probability Weighting
            \item Difference-in-Differences
            \item Synthetic Control
            \item Synthetic Difference-in-Differences
            \item Regression Discontinuity
        \end{itemize}
    \end{itemize}

    \item Choice Model
    \begin{itemize}
        \item Binary Choice Model
        \item Multinomial Choice - Ordered
        \item Multinomial Choice - Non-Ordered
        \item Nested Logit Model
    \end{itemize}

    \item Selection Model
    \begin{itemize}
        \item Tobit Model
        \item Others
    \end{itemize}

\end{itemize}

\end{document}
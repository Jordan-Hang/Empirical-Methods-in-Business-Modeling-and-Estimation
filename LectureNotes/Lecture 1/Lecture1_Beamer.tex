\documentclass[10pt]{beamer}
\usepackage{amsmath,amsfonts,amssymb}
\usepackage[utf8]{inputenc}
\usepackage{xeCJK}
\usepackage{graphicx}
\usepackage{mathtools}
\usepackage{geometry}
\usepackage{enumitem}
\usepackage{hyperref}
\usepackage{ragged2e}
\usepackage{setspace}
\setstretch{1.1}

% Customize the symbols
\setlist[itemize,1]{label=$\bullet$}
\setlist[itemize,2]{label=$\circ$}
\setlist[itemize,3]{label=$\triangleright$}

\title{Notes on Empirical Methods in Business \\ Lecture 1: Introduction and Research Classification}
\author{Hang XU}
\institute{hxuch@connect.ust.hk}
\date{\today}
\usetheme{Madrid}

\begin{document}

\frame{\titlepage}

% get a table page
\begin{frame}
\frametitle{Table of Contents}
\tableofcontents
\end{frame}

\section{Introduction}
\begin{frame}
\frametitle{Background}
\justifying
\begin{itemize}
    \item During the first year of my PhD, I have taken a course on Empirical Methods in Business: Modeling and Estimation taught by Prof. Tat Chan from WUSTL. 
    \item This course is designed to provide students with a comprehensive understanding of the most commonly used empirical methods in business research. 
    \item The main topics of the course can be seen in section 3. I mainly used handwritten notes when I was taking the lectures, which is hard to formalize. 
\end{itemize}
\end{frame}

\begin{frame}
\frametitle{Purpose of the Notes}
\justifying
\begin{itemize}
    \item As a PhD researcher specializing in empirical studies, it is necessary to have a clear understanding of common empirical methods. Therefore, I review the course content and summarize it in a more formal way to help others who are interested in empirical methods in business research.
    \item Worth to mention that, my notes are mainly based on what prof. Tat Chan's lecture notes, \textbf{but all the faults in this notes are mine}. I will try my best to make it accurate and clear. 
    \item If you find any mistakes or have any suggestions, please feel free to contact me.
\end{itemize}
\end{frame}


\section{Research Classification}
\begin{frame}
\frametitle{Research Classification}
Traditional classifications in empirical research:
\begin{itemize}
    \item Controlled data: Lab, AFE, FFE
    \begin{itemize}
        \item Field experiment: AFE (artefactual field experiment), FFE (framed field experiment)
        \item Lab experiment
    \end{itemize}
    \item Naturally occurring / observational data
    \begin{itemize}
        \item Natural experiment: NE, NFE (natural field experiment\footnote{Field experiment that happens naturally, and people do not realize the experiment they are in.})
        \item Market data: IV, PSM, STR (Structural modeling)
    \end{itemize}
\end{itemize}
\end{frame}

\begin{frame}
\frametitle{Causal Treatment Effects}
Identify the \underline{causal treatment effects} has been the main focus of empirical research in business.
\begin{itemize}
    \item The golden rule for identification: \textbf{Randomization of treatment status}.
    \begin{itemize}
        \item $y_i = \alpha + \gamma T_i + \epsilon_i$, where $T_i$ is the treatment status.
        \item Randomization makes $E(\epsilon_i | T_i = 0) = E(\epsilon_i | T_i = 1)$.
        \item Thus, $\gamma$ can identify the causal effect of treatment.
    \end{itemize}
    \item No endogenous issues: 
    \begin{itemize}
        \item People cannot quit or switch the groups.
        \item No spillover effect:
        \begin{itemize}
            \item Across sides: two-sided platform, sellers and buyers switch -- no reverse causality;
            \item Across groups in one side: individuals in each group do not aware they are treated or controlled. i.e., no information spillovers.
        \end{itemize}
    \end{itemize}
\end{itemize}
\end{frame}

\begin{frame}
\frametitle{Identifying Causal Effects with Market Data}
Market data cannot be randomized, so we need to use other methods to identify the causal effect of treatment:
\begin{itemize}
    \item Statistical methods: Approximating the experiments: e.g., DiD
    \item Econometric methods:
    \begin{itemize}
        \item Control methods
        \item Instrument variables
        \item Structural models
    \end{itemize}
\end{itemize}
\end{frame}

\begin{frame}
\frametitle{Pay Attention to Data and Assumptions}
\begin{itemize}
    \item Many researchers focus more on fancy methods, ignoring the data and assumptions, making the story less reliable.
    \item Questions need to think before digging into the research:
    \begin{itemize}
        \item What is the data? Can it help identify the causal effects?
        \item What are the identification assumptions? Are they reasonable?
    \end{itemize}
\end{itemize}
\end{frame}

\begin{frame}
\frametitle{Key Components in Empirical Research}
4 key components in empirical research:
\begin{itemize}
    \item \textbf{Research Questions}
    \begin{itemize}
        \item Why are your research questions important?
        \item What is the use for business/consumers/regulators?
        \item What is your contribution to the literature?
    \end{itemize}
    
    \item \textbf{Data}
    \begin{itemize}
        \item Can your data help address your research questions?
    \end{itemize}
    
    \item \textbf{Model}
    \begin{itemize}
        \item What is \(Y\)? What are your \(X\)'s?
        \item What is the relationship between \(Y\) and \(X\)'s?
        \item What is the data generating process (DGP)?
        \item How does your model address your research questions?
    \end{itemize}
    
    \item \textbf{Estimation}
    \begin{itemize}
        \item OLS / NLS? MLE? Method of moments? Other advanced methods?
        \item What is the identification of model parameters?
    \end{itemize}
\end{itemize}
\end{frame}

\begin{frame}
    \frametitle{Main Content of the Notes}
    \begin{itemize}
        \item In this note, I will focus more on \textbf{modeling} and \textbf{estimation}. 
        \item Given the research question and data, how to build up the model, what are potential issues of the model, and how to estimate the parameters are the interests.
        \item What is a model?
        \begin{itemize}
            \item A general model: $Y = f (X, e; \beta)$
            \item Specification: how to define $f(\cdot)$ and the distribution of $e$
            \item Effect of $X$ on $Y$: $\beta$
            \begin{itemize}
                \item $Y$: Interested outcome variable.
                \item $X$: Important business policies / actions + controls.
            \end{itemize}
        \end{itemize}
        \item Main challenge: Can we identify true $\beta$ from the data by using appropriate estimation methods? 
    \end{itemize}
\end{frame}


\section{Main Methods Covered}
\begin{frame}
    \frametitle{Topic 1: Regressions}
    The main topics covered in the course:
    \begin{itemize}
        \item Issues in Regressions
        \begin{itemize}
            \item Specification
            \item Multicollinearity
            \item Heteroskedasticity
            \item Endogenity
        \end{itemize}
        \item Endogenity Solutions
        \begin{itemize}
            \item Instrument Variables
            \item Panel Data with Fixed Effects
        \end{itemize}
    \end{itemize}
\end{frame}

\begin{frame}
    \frametitle{Topic 2: Treatment Effects and Causal Inference}
    \begin{itemize}
        \item Treatment Effect and Causal Inference
        \begin{itemize}
            \item Introduction Treatment Effects
            \item Causal Inference Methods
            \begin{itemize}
                \item Matching
                \item Propensity Score Matching
                \item Inverse Probability Weighting
                \item Difference-in-Differences
                \item Synthetic Control
                \item Synthetic Difference-in-Differences
                \item Regression Discontinuity
            \end{itemize}
        \end{itemize}
    \end{itemize}
\end{frame}

\begin{frame}

\frametitle{Topic 3: Advanced Methods and Structural Modeling}
\begin{itemize}
    \item Choice Model
    \begin{itemize}
        \item Binary Choice Model
        \item Multinomial Choice - Ordered
        \item Multinomial Choice - Non-Ordered
        \item Nested Logit Model
        \item Others
    \end{itemize}

    \item Selection Model
    \begin{itemize}
        \item Tobit Model
        \item Others
    \end{itemize}
\end{itemize}
\end{frame}


\end{document}
